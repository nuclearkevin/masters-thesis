\chapter*{Abstract}
\addcontentsline{toc}{chapter}{Abstract}
\setheader{Abstract}

The next generation of nuclear reactors pose a large challenge to existing computational methods. \texttt{Caribou} is a MOOSE-based health physics and environmental impact code under development at Ontario Tech which aims to address some of these challenges. This work developed a discrete ordinates radiation transport solver and a radionuclide trace species transport solver in the MOOSE framework for \texttt{Caribou}. This radiation transport solver is compared to several benchmark problems to determine its accuracy with and without ray effect mitigation measures, finding good agreement with all the problems tested. The trace species transport solver is then verified with the method of manufactured solutions. The coupled solvers are then used to analyze the formation of $\mathrm{^{41}Ar}$ in a containment volume due to ex-core neutron fields, and the photon fields from a $\mathrm{^{137}Cs}$ plume. The conclusions of this work indicate that these methods are a valuable addition to \texttt{Caribou}'s suite of capabilities.

\bigskip
\bigskip
\bigskip
\bigskip

\noindent \textbf{Keywords:} Ex-Core Radiation Transport; Mobile Depletion; Discrete Ordinates; MOOSE; \texttt{Caribou}