\chapter{Recommendations for Future Work} 
\label{recommendations}

The remainder of this thesis focuses on providing recommendations for future work on the coupled radiation transport and mobile depletion solvers implemented. This includes recommendations for additional solver features, performance enhancements, and verification and validation. 

Several improvements to the \acrshort{sn} radiation transport solver can be made to enhance its performance, allowing for the use of additional energy groups, angular directions, and a more refined mesh. The first performance improvement that should be made is the implementation of Gauss-Seidel iteration and scattering source iteration \cite{denovo,computational_methods}. This results in the decoupling of all energy groups and all directions with an inner/outer iteration scheme, allowing for all directions to be solved for independently. The current scattering implementation in the \acrshort{sn} solver has all of the directions for all energy groups solved in one sparse matrix without the Jacobian contribution for the scattering term; resulting in slow convergence and a large memory penalty. The second recommended performance improvement would be the addition of nonlinear diffusion acceleration \cite{cgfem_saaf_sn_2} and one-group transport acceleration for Gauss-Seidel iteration \cite{denovo}. This would speed up the convergence of scattering source iteration and Gauss-Seidel iteration, respectively. The particle diffusion equation is unsuited for the void and near-void regions that \acrshort{gnat} solves radiation transport in, and so non-local diffusion coefficients \cite{non_local_morel_larsen} would be required for the diffusion solver used for acceleration. As far as the author can tell, non-local diffusion theory has not been investigated for problems which have substantial void regions such as containment volumes; this may prove to be an additional area of study. 

There are several enhancements that can be made to the uncollided flux techniques described in this work. The ray traced uncollided flux method requires a large number of rays and a very fine mesh resolution owing to the calculation of the average flux in a cell, which is a first order approximation of the continuous solution. This can be mitigated by tracing rays to the nodes of a discontinuous linear Lagrange (or higher order) basis function across the element, then solving a system of equations to obtain the coefficients required to represent the uncollided flux across the element \cite{harbour_uncollided}. An additional method which is worth considering is the current state of the art for ray traced uncollided flux technques: ray splitting for arbitrary finite element meshes \cite{fem_arbitrary_uncollided}. This approach results in a several order of magnitude reduction in the number of rays required to compute the uncollided flux when compared to methods that use a direct quadrature with comparable accuracy to those methods. Out of the two implemented ray effect mitigation technique the \acrshort{sasf} is the most preliminary, but shows a large amount of promise. The technique needs to be adapted to function for surface sources and volumetric sources, which can be done by using a numerical quadrature over these sources. A disadvantage of the \acrshort{sasf} approach is the need for the user to define a near-source region and properly implement said region into the provided mesh. This disadvantage could be removed through the use of the \acrshort{moose} mesh generator suite to cut out either a spherical or quadrilateral near-source region from the mesh and automatically generate the near-source region boundary. 

There is a vast amount of future work available when it comes to the verification and validation of these solvers. The radiation transport suite was more thoroughly verified then the mobile depletion solvers in this work; future studies should aim to further verify the mobile depletion solvers over the radiation transport solver. This could be done through benchmark problems or analytical solutions. Comparisons with the codes used by the fusion community such as GammaFlow \cite{fusion_activation_wall} or FLUNED \cite{fusion_activation_tool_fluned} also present a possible opportunity for future verification. In addition to the mobile depletion solvers the coupling between the two different physics needs to be verified. The verification of multiphysics methods is still an active area of research and so the options for performing verification studies are limited. The use of the \acrshort{mms} for both of the solvers coupled together may be an option, alongside code to code comparisons. When it comes to validation, there is a lack of experimental benchmarks which can be used for this variety of coupled work. The most promising open benchmark for validating the coupling between neutronics and mobile depletion is the International Thermonuclear Experimental Reactor first wall mock-up at the Frascati Neutron Generator facility \cite{fusion_activation_wall}. Validation of the coupled methods for dispersion dose rate calculations can be performed using the radiological dispersion device field trials conducted in Suffield, Alberta \cite{nf_rdd_trials_i, nf_rdd_trials_ii}. When it comes to validating the solvers separately, the SINBAD radiation shielding benchmark suite \cite{sinbad} could be used. Experiments that track a radioactive tracer and its progeny would work for validating mobile depletion on its own. 