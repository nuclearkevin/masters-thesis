\chapter{Conclusions} 
\label{conclusion}

There are several challenges in modelling and simulation regarding Generation IV and \acrshort{smrs}, some of these challenges include: additional scrutiny placed on existing analysis methodologies; novel source terms due to different reactor spectra; smaller containment systems coupled with the larger leakage fractions resulting in higher relative ex-core radiation fields; and more complex reactor components (such as the piping in \acrshort{msr}s) which result in more deep penetrations. \texttt{Caribou} is a response to these health physics and environmental impact challenges of advanced reactors, and this work contributed a series of new capabilities to \texttt{Caribou} which aim to address the subset of these challenges which involve radiation transport and mobile depletion (trace species transport for nuclides in a radiation field).

The main result of this work is the development of a coupled radiation transport and mobile depletion solver named \acrshort{gnat} for the health physics and environmental impact code \texttt{Caribou}. The developed radiation transport solver is based on the \acrfull{sn} method of angular discretization and \acrlong{cgfem} with the \acrlong{saaf} approach for spatially discretizing the linear Boltzmann transport equation. The main limitation of the \acrshort{sn} method is the presence of numerical oscillations in the solution for problems that contain large optically thin regions, which are known as ray effects. To mitigate ray effects, two approaches for computing the uncollided particle flux moments from sources were implemented in the radiation transport solver: the method of ray tracing and the \acrfull{sasf} method. The \acrshort{sasf} method is a novel approach developed in this work. A mobile depletion solver was then implemented using two spatial discretization approaches. The first is the \acrlong{cgfem} using \acrlong{supg} stabilization. The second is the \acrlong{fvm} using closure statements provided by \acrshort{moose}. These solvers were verified and then used to analyze two demonstration problems: the formation of the radionuclide effluent $\mathrm{^{41}Ar}$ in a small reactor containment building and the scalar photon fluxes from a $\mathrm{^{137}Cs}$ plume. Below is a list of general conclusions regarding the viability of this multiphysics solver:
\begin{itemize}
    \item The \acrshort{sn} radiation transport solver is capable of converging to a simple analytical solution with a strong heterogeneity as the angular and spatial domain was refined. The \acrshort{sn} solver performed well in all of the Kobayashi benchmark problems, where the maximum error can be primarily attributed to either ray effects in the case of the second Kobayashi benchmark or a poorly refined mesh in the case of the first Kobayshi benchmark problem. 
    \item While not designed for this purpose, the radiation transport solver is suitably accurate for performing criticality eigenvalue calculations across a variety of energy group structures. A consistent bias on $k_{eff}$ in the range of $-214.5$ pcm to $273.9$ pcm (depending on the benchmark case) indicates that any inconsistencies are likely a result of the coarse mesh and low order angular quadrature set. Additional verification should be performed with a fine mesh and high order angular quadrature set to confirm this conclusion.
    \item The ray traced uncollided flux method is reasonably accurate and capable of removing ray effects from the \acrshort{sn} transport solver with a modest increase in computational complexity when compared to refining the angular quadrature set. 
    \item The \acrshort{sasf} approach is one of the novel contributions of this work and proves to be a valuable area of future study. The use of this method to analyze a modified version of the third Kobayashi benchmark showed that the technique is free of any ray effects and matches the ray traced reference well over the majority of the computational domain. 
    \item The mobile depletion solvers were found to have the anticipated order of convergence in both space and time for multiple basis functions (finite element) or closures (finite volume). This indicates that the implementation of the base capability of trace species transport is correct.
    \item Neutron transport methods informing mobile depletion prove to be invaluable to predicting the distributions of fluid-borne radionuclides in containment systems. The first demonstration problem showcases how these methods allow for the prediction of localized patches of high activity within containment structures, which may be of use for radiation protection in outage scenarios and to better predict effluent emissions from containment systems.
    \item Mobile depletion coupled with photon transport may be useful for predicting the dose rates from radionuclide plumes. The second demonstration problem showed that these methods predict sky shine effects and shadowing behind buildings without requiring additional effort from the analyst. If the appropriate geometry were to be modelled, this approach would be capable of modellings scattering off of buildings and the ground as well.
\end{itemize}

It is important to state that the main contributions of this work were the development of the radiation transport and mobile depletion solvers, the novel ray effect mitigation method, the verification of the solvers, and the application of these capabilities to problems in health physics through the demonstration problems. There is a substantial amount of future work which may be performed to enhance this solver, and further verify and validate the approach taken. This future work is the subject of Chapter~\ref{recommendations}.